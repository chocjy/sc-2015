\section{Setup}
\label{sec:setup}

\subsection{Platforms}

  \begin{table*}
    \begin{center}
    \begin{tabular}{| l | c | c | c | c | c | c | c |}
    \toprule
    \textbf{Platform} & \textbf{Total Cores} & \textbf{Core Frequency} & \textbf{Interconnect} & \textbf{DRAM} & \textbf{SSDs} \\
    \midrule
    Amazon EC2 r3.8xlarge & 960 (32 per-node) & 2.5 GHz & 10 Gigabit Ethernet & 244 GiB & 2 x 320 GB \\
    \midrule
    Cray XC40 & 960 (32 per-node) & 2.3 GHz & Cray Aries\texttrademark & 252 GiB & None \\
    \midrule
    Experimental Cray cluster & 960 (24 per-node) & 2.5 GHz & Cray Aries\texttrademark & 126 GiB & 1 x 800 GB \\
    \bottomrule
    \end{tabular}
    \end{center}
    \caption{Specifications of the three hardware platforms used in these performance experiments.}
    \label{tab:hwspecs}
  \end{table*}
  
 We evaluated the performance of CX matrix factorization of three platforms:
 \begin{itemize}
 \item a Cray XC40,
 \item an experimental Cray cluster, and
 \item an Amazon EC2 r3.8xlarge instance.
 \end{itemize}
 For all platforms, we sized the Spark job to use 960 executors cores (except as otherwise mentioned).  Table~\ref{tab:hwspecs} shows the full specifications of the three platforms.
  
\subsection{MSI Dataset}
\paragraph{Mass spectrometry imaging with ion-mobility}
Mass spectrometry measures ions that are derived from the molecules present in a complex biological sample.
These spectra can be acquired at each location (pixel) of a heterogeneous sample, allowing for collection of spatially
resolved mass spectra.
This mode of analysis is known as \textit{mass spectrometry imaging (MSI)}.
The addition of \textit{ion-mobility separation (IMS)} to MSI adds another dimension, drift time, to an already multidimensional
dataset ($x$ location, $y$ location, and $m/z$).
IMS improves sensitivity for low-abundance compounds and can separate ions that have the same mass but which have
different chemical structures (isomers).
The rate of collision of ions with carrier gas molecules depends on the molecule's shape, not just m/z.
Thus, the transit time of ions passing through a tube filled with ``high'' pressure carrier gas is slower
for ions with high collisional cross section and faster for those that collide less frequently, allowing separation of
isomeric ions.
These two-dimensional spectra (drift-time, m/z) can be measured at each location (x, y) in a
heterogeneous biological sample.
The combination of IMS with MSI is finding increasing applications in the study of disease diagnostics, plant
engineering, and microbial interaction

\paragraph{OpenMSI as a ``model'' for large-scale cloud analytics}
OpenMSI is a tool for web-based visualization, analysis and
management of mass spectrometry imaging (MSI) data~\cite{OpenMSI}.  MSI is being increasingly applied to image complex samples for
applications spanning health, microbial ecology, and high throughput screening in drug research and synthetic biology.
MSI has emerged as a tool of choice for applications where understanding the spatial variation of metabolism is
critical. Unfortunately, the scale of MSI data and complexity of analysis presents a significant challenge to
scientists: a single 2D-image may be many gigabytes and comparison of multiple images is beyond the capabilities
available to many scientists. The OpenMSI project surmounts these challenges by providing (i) a web-based gateway for
management and storage of MSI data, (ii) tools for facile visualization of fundamentally hyper-dimensional MSI datasets,
and (iii) interfaces to specialized and general tools for advanced statistical analysis of MSI data.

\paragraph{Utility of CX/PCA in MSI}
Typical mass spectrometry imaging (MSI) experiments generate large volumes of data, particularly when combined with new
technologies that provide even more dimensionality, such as ion mobility separation.
Dimensionality reduction techniques can help reduce the data down to a manageable, approachable size.
Typical approaches for dimensionality reduction include PCA and NMF, but interpretation of the results is difficult
because components or factors are significant functions of many hundreds or thousands of features in the original data.
CX is a dimensionality reduction technique that circumvents this problem by identifying small numbers of columns in the
original data that reliably explain a large portion of the variation in the data.
In MSI, this allows rapidly pinpointing important ions and locations.

